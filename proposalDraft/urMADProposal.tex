\documentclass{article}

\usepackage{url}
\usepackage{anysize}
\marginsize{2cm}{2cm}{2cm}{2cm}
\usepackage{xcolor}
\usepackage{array}

\newcommand{\todo}[1]{\textcolor{red}{#1}}

\setlength{\parindent}{0cm}
\setlength{\parskip}{0.2cm}

\title{\texttt{urMAD}\hspace{0.5cm} Grand Challenge ICMI'12}
\author{\textbf{u}nification and \textbf{r}eformation of
 \textbf{M}ultimodal \textbf{A}nnotations of \textbf{D}atasets}
\date{}

\begin{document}
 \maketitle
 \hrule
 \vspace{0.1cm}
 \hrule
 \tableofcontents
 \vspace{0.5cm}
 \hrule
 \vspace{0.1cm}
 \hrule

 \section{Preface}
 \label{sec:preface}
I would expect that people could be split in three groups after reading this document according to their a posteriori
feelings:
\begin{description}
 \item [indifferent] Of course. Even if this is a hot topic, not everybody will be interested on it. \textit{What to
do?} Please forward this document to the colleagues of yours that may be interested.
 \item [excited] Great! We want you to collaborate with us in this project. \textit{What to do?} Please, contact us. We
will keep you updated!
 \item [angry] Because (part of) the project has been already done by you/your colleagues. \textit{What to do?} Please
let us know! This is a proposal and we are willing to make it consistent with the state-of-the-art and sound. Your
comments and feedback are really valuable to us.
\end{description}


 \section{Aim}
Walk towards a unified criteria for evaluation of multimodal annotation of data sets.

 \section{Introduction}
 The interest of the research community in the field of multimodal interaction has been growing during the last decades.
New sensors (haptics, depth, high resolution cameras, microphone arrays, accelerometers) have been developped and some
of them have been commercialized as consumer sensors (e.g. Kinect). This speeded up the research on multimodal
interaction and pushed the developpement of new algorithms to release software able to deal with combinations of such
sensors.

As a natural fact, researchers are interested in different problems and applications. That is why, the current
multimodal datasets are designed, annotated and evaluated depending on the task they should be useful to solve.
Regardless the application targeted, the data may be worth to other researchers approaching other type of problems.
That is why, we propose the \texttt{urMAD} Grand Challenge. As stated before, the aim of \texttt{urMAD} is the genesis
of a common framework for description, annotation and evaluation of multimodal datasets.

 \section{Tasks}
The main task is to evaluate annotation in multimodal data sets. To reach this goal we need:
\begin{description}
 \item [Data sets] The should be, of course, multimodal: so data acquired with different kind of sensors. I would like
our work to be visible and useful, so I would propose to work ONLY with publicly available data sets. How do you feel
about it?
 \item [Targetted annotations] For each data set, we should have which annotations are we targetting (gestures, speech,
head nods, ...)
 \item [Evaluation metrics] I think we should predefine which criteria we would like people use to evaluate their
annotations. I also think that we should not stay with these criteria, but allow people to propose new ones.
\end{description}

 \section{Annotation format and tools}
I would like to point out that its worth to push (even force) people to make their annotation formats public. We could
even specify the annotation format from the challenge. Otherwise the rest of the world won't be able to use it. It would
be also nice it the annotation tools are also open source, although I think this is much more difficult to reach :).

 \section{Procedure}
We have (at least) two ways to proceed:
\begin{itemize}
 \item Choose some existing data sets and what to annotate on them. Then set up the challenge for ICMI'12 in the form
of workshop with the aim to have nice way to evaluate multimodal annotation.
 \item Set the challenge in two steps. During the first step we will be in charge of recording a new multimodal data
set. Be careful: this is a lot of work! We have to design it and acquire it. However, we will have complete control on
the contents and on the recording devices. During the second step (may be in two years) we will set up the workshop as
in point 1.
\end{itemize}


 \section{Format}
 We need to choose a format for the \texttt{urMAD} Grand Challenge and to reformat the proposal according to the call
for Challenges (\url{http://www.acm.org/icmi/2012/index.php?id=cfc}), that is:
\begin{itemize}
 \item Title
 \item Abstract appropriate for possible Web promotion of the Challenge
 \item Detailed description of the challenge and its relevance to multimodal interaction.
 \item Plan for soliciting participation
 \item Proposed schedule for releasing datasets and receiving submissions.
 \item Short bio of the organizers
 \item Funding source (if any) that supports or could support the challenge organization.
 \item Preference (if any) for special session or workshop format.
\end{itemize}
with a maximum number of five pages.

 \section{Contact}
 Right now there is just one contact person.\vspace{0.2cm}\\
 Xavier Alameda-Pineda, find me at: \url{xavier.alameda-pineda@inria.fr}\vspace{0.2cm}\\
 I hope people will be interested in such Grand Challenge and we will create a nice working group. It would be nice to
have support resources, like SVN repositories, wikies, ... I am working on it, at least to be able to collaboratively
write the proposal.\vspace{0.3cm}\\
Thank you for reading. Please do not forget to re-read Section \ref{sec:preface}.

%   \section*{Notes}
%   Contact the guy of: \url{http://www.clear-evaluation.org}
% 
% ELAN: \cite{Sloetjes2008} A nice framework to annotate multmimedia data. Also trying to work towards the DCR of
% the ISO.
% 
% \bibliographystyle{plain}
% \bibliography{../BibliographyMendeley/AnnotationTools}

\end{document}
