\documentclass{sig-alternate}

% The following packages can be found on http:\\www.ctan.org
\usepackage{graphicx} % for pdf, bitmapped graphics files
%\usepackage{epsfig} % for postscript graphics files
%\usepackage{mathptmx} % assumes new font selection scheme installed
%\usepackage{times} % assumes new font selection scheme installed
\usepackage{amsmath} % assumes amsmath package installed
\usepackage{amssymb}  % assumes amsmath package installed
\usepackage{url}
\usepackage{float}
% \usepackage{cite}
% \usepackage{math}
\usepackage{array}
\usepackage{multirow}

\usepackage{algorithm}
\usepackage{algorithmic}

\usepackage{times}
\usepackage{helvet}
\usepackage{courier}

\usepackage[small,bf]{caption}

\usepackage{color}
\definecolor{darkgreen}{rgb}{0.0,0.5,0.0}
\definecolor{red}{rgb}{1.0,0.0,0.0}

\newcommand{\todo}[1]{ \textcolor{red}{\bf #1}}

\usepackage{algorithm}
\usepackage{algorithmic}

\usepackage[caption=false,font=footnotesize]{subfig}
\usepackage{multirow}
\usepackage{ifpdf}

\begin{document}


\title{D-META Grand Challenge}
\subtitle{Data sets for Multimodal Evaluation of Tasks and Annotations}
% \subtitle{}
% %
% % You need the command \numberofauthors to handle the 'placement
% % and alignment' of the authors beneath the title.
% %
% % For aesthetic reasons, we recommend 'three authors at a time'
% % i.e. three 'name/affiliation blocks' be placed beneath the title.
% %
% % NOTE: You are NOT restricted in how many 'rows' of
% % "name/affiliations" may appear. We just ask that you restrict
% % the number of 'columns' to three.
% %
% % Because of the available 'opening page real-estate'
% % we ask you to refrain from putting more than six authors
% % (two rows with three columns) beneath the article title.
% % More than six makes the first-page appear very cluttered indeed.
% %
% % Use the \alignauthor commands to handle the names
% % and affiliations for an 'aesthetic maximum' of six authors.
% % Add names, affiliations, addresses for
% % the seventh etc. author(s) as the argument for the
% % \additionalauthors command.
% % These 'additional authors' will be output/set for you
% % without further effort on your part as the last section in
% % the body of your article BEFORE References or any Appendices.
%

\numberofauthors{3}
\author{
\alignauthor Xavier Alameda-Pineda \\
\affaddr{INRIA Grenoble Rh\^one-Alpes,} 
\affaddr{655, Av. Europe, 38334 Montbonnot,} 
\affaddr{University of Grenoble, France} \\
\email{xavier.alameda-pineda@inria.fr}
% Adjunct Professor of Language Technology
% University of Helsinki
\alignauthor{Dirk Heylen} \\
\affaddr{Human Media Interaction,} 
\affaddr{PO BOX 217, 7500 AE Enschede,}
\affaddr{University of Twente, The Netherlands}\\
\email{d.k.j.heylen@utwente.nl}
\alignauthor Kristiina Jokinen \\
\affaddr{Department of Behavioural Sciences,}
\affaddr{PO BOX 9, FIN-00014,}
\affaddr{University of Helsinki, Finland}\\
\email{kristiina.jokinen@helsinki.fi}
} 


% \conferenceinfo{ICMI'11,} {November 14--18, 2011, Alicante, Spain.}
% \CopyrightYear{2011}
% \crdata{978-1-4503-0641-6/11/11}
% \clubpenalty=10000
% \widowpenalty = 10000

\maketitle
%%%%%%%%%%%%%%%%%%%%%%%%%%%%%%%%%%%%%%%%%%%%%%%%%%%%%%%%%%%%%%%%%%%%%%%%%%%%%%%%
\begin{abstract}
aasdf
\end{abstract}
\section{Introduction}

\section{Description}
The ultimate goal of the \texttt{D-META} Grand Challenge is to define a framework unifying the annotations of multimodal
data sets. This framework would be extremely useful, since researchers all over the world would be able
to compare their algorithms, assuming some standard annotation. At the same time, they would annotate their data sets
under this framework and provide to these data sets world-wide visibility.

Unfortunately, we are far from this goal because of several reasons (\todo{rephrase and cite})
\begin{itemize}
 \item annotation depends on application target
 \item exist conceptual differences between researchers
 \item such framework should take into account all kind of sensors, which is already very difficult due to the existing
variety of sensors and the creation of new sensors
\end{itemize}

This is why we believe that such a framework does not have to be imposed to (accepted by) the community, but built from
the community little by little.


\section{Participation plan and schedule}

\section{Organizers' bio}
\subsection{Xavier Alameda-Pineda}
\subsection{Kristiina Jokinen}
\subsection{Dirk Heylen}

\section{Funding}

\section{Format}

\end{document}
